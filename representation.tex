\section{Repr�sentation}
\subsection{Isomorphismes accidentels}

\begin{frame}[fragile]
  \begin{block}{Isomorphismes accidentels}
    $$Spin(1) = O(1)$$
    $$Spin(2) = U(1) = O(2)$$
    $$Spin(3) = Sp(1) = SU(2)$$
    $$Spin(4) = Sp(1) � Sp(1)$$
    $$Spin(5) = Sp(2)$$
    $$Spin(6) = SU(4)$$
  \end{block}
\end{frame}

%% \subsection{Description d�taill�e}
%% 
%% \begin{frame}
%%   \begin{block}{Dimension 1}
%%     En dimension 1, la repr�sentation d'un spineur est, de mani�re formelle,
%%     Majorana, un r�el unidimensionnel.
%%   \end{block}
%% 
%%   \begin{block}{Dimension 2}
%%    En deux dimensions (dans un espace Euclidien), la partie gauche et droite
%%     du spineur de Weyl repr�sentent des composantes complexes, c'est  � dire
%%     des nombres complexes qui peuvent �tre multipli�s par $e^{\pm i\phi/2}$
%%     pour r�aliser une rotation d'angle $\phi$.
%%   \end{block}
%% \end{frame}
%% 
%% \begin{frame}
%%   \begin{block}{Dimension 3}
%%     En trois dimensions, la repr�sentation est bidimensionnelle et
%%     pseudo-r�elle; L'existence de spineurs en dimension 3 est
%%     induit par l'isomorphisme
%%     des groupes $SU(2) \cong \mathit{Spin}(3)$ qui permettent de d�finir
%%     l'action de Spin(3) sur un complexe � deux composantes colonnes;
%%     le g�n�rateur de SU(2) peut �tre �crit sous la forme de matrices de
%%     Pauli.
%%   \end{block}
%% 
%%   \begin{block}{Dimension 4}
%%     En dimension 4, l'isomorphisme correspondant est
%%     $Spin(4) \equiv SU(2) \times SU(2)$.
%%     Il y a deux composantes des spineurs de Wayl pseudo-r�els diff�rencies et
%%     chacun d'eux peut se factoriser en �l�ments de SU(2).
%%   \end{block}
%% \end{frame}
%% 
%% \begin{frame}
%%   \begin{block}{Dimension 5}
%%     En dimension 5, l'isomorphisme adapt� est
%%     $Spin(5)\equiv USp(4)\equiv Sp(2)$ qui implique que la repr�sentation
%%     est de dimension 4 et pseudo-r�elle.
%%   \end{block}
%% 
%%   \begin{block}{Dimension 6}
%%     En dimension 6, l'isomorphisme $Spin(6)\equiv SU(4)$
%%     garantit qu'il y a deux repr�sentations complexes de Weyl
%%     de dimension 4 qui sont leurs conjugu�s r�ciproques.
%%   \end{block}
%% \end{frame}
%% 