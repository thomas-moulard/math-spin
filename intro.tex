\section{Introduction et d�finition du groupe Spin}
\subsection{Pourquoi?}

\begin{frame}
  \begin{block}{Quel est le but de cette pr�sentation?}
    \begin{itemize}
    \item Introduire le groupe Spin (pr�sentation g�n�rale)
    \item ...et expliquer le lien avec son interpr�tation physique.
    \end{itemize}
  \end{block}
\end{frame}


\subsection{Groupe Spin}

\begin{frame}
  \begin{block}{D�finition du groupe Spin}
    En math�matiques, le groupe Spin de degr� n, not� Spin(n),
    est un rev�tement double particulier du groupe sp�cial orthogonal
    r�el $SO(n,\mathbb{R})$. C'est-�-dire qu'il existe une suite exacte
    de groupes de Lie:

    $$1\to\mathbb{Z}_2\to\operatorname{Spin}(n)\to\operatorname{SO}(n)\to 1$$


    Pour $n > 2$, Spin(n) est simplement connexe et co�ncide avec le
    rev�tement universel de $SO(n,\mathbb{R})$.
    En tant que groupe de Lie, Spin(n) partage sa dimension $n(n-1)/2$
    et son alg�bre de Lie avec le groupe sp�cial orthogonal.


    Spin(n) peut �tre construit comme un sous-groupe des �l�ments
    inversibles de l'alg�bre de Clifford C(n).

  \end{block}
\end{frame}

\begin{frame}
  \begin{block}{D�finition d'un rev�tement}
    En math�matiques, et plus particuli�rement en topologie,
    un rev�tement d'un espace topologique X par un espace topologique C
    est une application continue et surjective
    $$p : C \rightarrow X$$
    telle que tout point $x\in X$ admette un voisinage ouvert U tel
    que l'image r�ciproque de U par p soit une union disjointe d'ouverts
    de C, chacun hom�omorphe � U par p.


    Il s'agit d'un cas particulier de fibration, � fibre discr�te.
  \end{block}
\end{frame}

\begin{frame}
  \begin{block}{Cardinalit� d'un rev�tement}
    Pour chaque $x \in X$, la fibre sur x est un sous-ensemble de C.
    Sur chaque composent connect� de X, la cardinalit� des fibres est la m�me
    (et possiblement infinie). Si chaque fibre poss�de deux �l�ment, on parle
    de rev�tement double.
  \end{block}
\end{frame}

\begin{frame}
  \begin{block}{Suite exacte}
    On d�finit une suite exacte pour des ensembles munis
    de la structure de groupe (groupe, anneau, module, etc...)
    et des homomorphismes entre ces ensembles.
    On supposera que les ensembles �tudi�s sont des groupes additifs.

    Soient $\left( {G_i} \right)_{i \in \mathbb{Z}}$
    des groupes et
    $f_i : G_{i} \rightarrow G_{i+1}$
    des morphismes de groupes, on dit que la suite:
    $$\ldots \rightarrow G_i \rightarrow G_{i+1}
      \rightarrow G_{i+2} \rightarrow ...$$
      est exacte si pour tout i on a $Im(fi) = Ker(fi + 1)$.
  \end{block}
\end{frame}