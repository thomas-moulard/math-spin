\section{Alg�bre de Clifford \&{} Groupe Spin}
%% wikipedia power :p

\def\Cl{\mathcal{C}\ell(V,Q)\,}
\def\Cln{\mathcal{C}\ell(V,Q)\,}
\def\Spinn{Spin\left(n\right)\,}
\def\SOn{SO_n\left(\mathbb R\right)\,}
\def\son{\mathfrak{so}\left(n\right)\,}
\def\Mnr{\mathcal{M}_n\left(\mathbb R\right)\,}
\newcommand{\Lie}[1]{Lie\left(#1\right)}

\subsection{Alg�bre de Clifford}
\begin{frame}
  \begin{block}{Produit Tensoriel}
    $$C_{ij} = A_i \otimes B_j$$
    $$\begin{bmatrix}
        a_1b_1 & a_2b_1 & a_3b_1 \\
	a_1b_2 & a_2b_2 & a_3b_2 \\
	a_1b_3 & a_2b_3 & a_3b_3 \\
	a_1b_4 & a_2b_4 & a_3b_4
      \end{bmatrix}
    = \begin{bmatrix}a_1 & a_2 & a_3\end{bmatrix}
      \otimes
      \begin{bmatrix}b_1 \\ b_2 \\ b_3 \\ b_4\end{bmatrix}
	$$
  \end{block}

  \begin{block}{Alg�bre de Clifford: Dimension}
    $$dim(\Cln) = 2^n$$
    $\Cln : $ Alg�bre de Clifford ($V$: espace vectoriel, $Q$: forme quadratique).
  \end{block}
\end{frame}

\subsection{Matrices Importantes}
\begin{frame}
  \begin{block}{Matrices de Pauli}
    \begin{center}
      $\sigma_x = \begin{bmatrix} 0 & 1 \\ 1 & 0 \end{bmatrix}$
      \ \ \ \ 
      $\sigma_y = \begin{bmatrix} 0 & -i \\ i & 0 \end{bmatrix}$
      \ \ \ \ 
      $\sigma_z = \begin{bmatrix} 1 & 0 \\ 0 & -1 \end{bmatrix}$
      \\
      Base de $SU(2)$.
    \end{center}
  \end{block}

  \begin{block}{Propri�t�s}
    $$\sigma_1^2 = \sigma_2^2 = \sigma_3^2 = \begin{pmatrix}
    1&0\\0&1\end{pmatrix} = I$$

    pour $i = 1, 2, 3$ :\\
    \begin{center}
      $\det (\sigma_i) = -1$ et $\operatorname{Tr} (\sigma_i) = 0$
    \end{center}

    \begin{center}
      $[\sigma_i, \sigma_j] = 2 i\,\epsilon_{i j k}\,\sigma_k$ et
      $\{\sigma_i, \sigma_j\} = 2 \delta_{i j} \cdot I$
    \end{center}
  \end{block}
\end{frame}

\def\sa{\sigma_x}
\def\sb{\sigma_y}
\def\sc{\sigma_z}

\begin{frame}
  \begin{block}{Matrice de Dirac}
    anticommutateur : $\left\{A,B\right\}=AB+BA$.
    \\
    Les matrices de Dirac: $\gamma_1,\ldots,\gamma_n$ engendre une
    alg�bre de Clifford. Ces matrices qui ont la propri�t� :
    $$\gamma_i\gamma_j + \gamma_j\gamma_i = 2\eta_{ij}$$
    o� $\eta$ est la matrice d'une forme quadratique.
  \end{block}
\end{frame}

\begin{frame}
  \begin{block}{Matrices de Dirac}
    \[ \begin{matrix}
      \gamma_1      &=& \sa \otimes  I  \otimes &\cdots &\otimes  I  \otimes  I  \\
      \gamma_2      &=& \sb \otimes  I  \otimes &\cdots &\otimes  I  \otimes  I  \\
      \gamma_3      &=& \sc \otimes \sa \otimes &\cdots &\otimes  I  \otimes  I  \\
      \gamma_4      &=& \sc \otimes \sb \otimes &\cdots &\otimes  I  \otimes  I  \\
                & \vdots &                      &\vdots &                        \\
      \gamma_{2k-3} &=& \sc \otimes \sb \otimes &\cdots &\otimes \sa \otimes  I  \\
      \gamma_{2k-2} &=& \sc \otimes \sb \otimes &\cdots &\otimes \sb \otimes  I  \\
      \gamma_{2k-1} &=& \sc \otimes \sb \otimes &\cdots &\otimes \sc \otimes \sa \\
      \gamma_{2k}   &=& \sc \otimes \sb \otimes &\cdots &\otimes \sc \otimes \sb \\
      \gamma_{2k+1} &=& \sc \otimes \sb \otimes &\cdots &\otimes \sc \otimes \sc
    \end{matrix} \]
    $$\gamma_{2k+1} = (-i)^k \gamma_1 \gamma_2 \, \cdots \,
    \gamma_{2k-1} \gamma_{2k}
    \quad
    \gamma_{i}^{2} = I
    $$
  \end{block}
\end{frame}

\subsection{Alg�bre de Lie de $\Spinn$}
\begin{frame}
  \begin{block}{Groupe $\Gamma$}
    Ensemble des �l�ments inversible de l'alg�bre de Clifford.
    $$\Gamma = \{ s \in \Cl | \forall x \in V,
    sxs^{-1} \in V \} $$
  \end{block}

  \begin{block}{Norme de Spin}
    $$\begin{matrix}
       Q: & \Gamma & \to \Gamma \\
	  & s      & \to \bar{s}s
      \end{matrix}$$
    involutions:
    $$\bar x = \alpha({}^tx) = {}^t\alpha(x)$$
    $$\mathcal{C}\ell^i(V,Q) = \{x \in \Cl | \alpha(x) = (-1)^ix\}\,$$
  \end{block}
\end{frame}

\begin{frame}
  \begin{block}{Groupe Spin}
    $$\begin{matrix}
      Pin &=& \{s \in \Gamma | Q(s) = \pm I\} \\
      Spin &=& Pin \cap \mathcal{C}\ell^0(V,Q)
      \end{matrix}$$
    Donc pour le groupe Spin on a:
    $$\bar x = \alpha({}^t\!x) = (-1)^0\cdot{}{}^t\!x = {}^t\!x$$
  \end{block}
\end{frame}

\begin{frame}
  \begin{block}{$\son \neq \SOn$}
    $$\son = \{ M \in \Mnr | {}^t\!M = -M \}$$
  \end{block}
  \begin{block}{$\Lie{\Spinn} \subseteq \son$}
    %% D�finition
    $$\Lie{\Spinn} = \{M \in \Mnr | \forall x \in \mathbb R,
    e^{xM} \in \Spinn\}$$
    %% equivalence relations
    $$e^{xM} \in \Spinn \Rightarrow Q(e^{xM}) = \pm I$$
    $$Q(e^{xM}) = {}^t\!e^{xM}e^{xM} = e^{x{}^t\!M}e^{xM}$$
    En d�rivant en $t = 0$ on obtient:
    $$\begin{matrix}
      Q(e^{tM}) = \pm I & \Rightarrow & {}^t\!M + M = 0\\
       & \Rightarrow & {}^t\!M = -M\\
       & \Rightarrow & \Lie{\Spinn} \subseteq \son
      \end{matrix}$$
  \end{block}
\end{frame}

\begin{frame}
  \begin{block}{$\Lie{\Spinn} = \son$}
    $\Spinn$ est un rev�tement de $\SOn$.\\[1ex]
    Donc $\Lie{\SOn}$ est isomorphe � $\Lie{\Spinn}$.\\[1ex]
    Ainsi $dim(\Lie{\Spinn}) = dim(\Lie{\SOn}) = dim(\son)$.\\[1ex]
    Or $\Lie{\Spinn} \subseteq \son$.\\[1ex]
    Donc $$\Lie{\Spinn} = \son$$
  \end{block}
\end{frame}
